\newpage
\begin{center}
    \bfseries \zihao{3} 摘~要
\end{center}
\par

近年来,随着生成式人工智能技术的快速发展,基于扩散模型的图像生成方法在计算机视觉领域展现出卓越的性能,
成为研究的热点。然而,这类模型通常需要处理复杂的多步骤计算过程,对硬件资源和实时性提出了更高要求。
以Stable Diffusion为例,其生成一张1024×1024分辨率的图像需经过几十次迭代,
单卡GPU的推理延迟较高,难以满足实际应用场景的需求。在此背景下,
许多研究引入分布式推理方法来降低推理延迟,以加速模型推理。
然而,现有的分布式推理方案依赖传统并行策略(如张量并行、序列并行)和机械化的异步通信机制。
这些方法虽然能实现基本的并行化,但存在两方面的缺陷:一方面,对高算力GPU集群的依赖导致硬件成本很高;
另一方面,频繁的跨设备通信引入了超过30\%的额外开销。
针对这一问题,本文提出一种基于特征相似度的自适应通信优化方法。通过量化分析相邻扩散步骤间特征的相似度
,设计了动态冻结关键值(KV)传输的通信策略,并使用异步传输方式降低通信开销。实验结果表明,该方法在保证图像生成质量
的前提下,大幅降低分布式推理延迟(最高可达33\%),为扩散模型分布式推理在低成本商用GPU集群上的高效部署提供
了解决方案。

\noindent \textbf{关键词:} 扩散模型,分布式推理,KV缓存,通信优化

\newpage
\begin{center}
    \bfseries \zihao{3} Abstract
\end{center}

\par
In recent years, with the rapid development of generative artificial intelligence technologies, 
image - generation methods based on diffusion models have demonstrated remarkable performance 
in the field of computer vision and become a research focus. However, such models typically 
need to handle complex multi - step computational processes, imposing higher requirements 
on hardware resources and real - time performance. Take Stable Diffusion as an example, 
generating an image with a resolution of 1024×1024 requires dozens of iterations, and the 
inference latency of a single - GPU card is relatively high, making it difficult to meet the 
demands of practical application scenarios. Against this backdrop, many studies have introduced 
distributed inference methods to reduce inference latency and accelerate model inference.
However, existing distributed inference schemes rely on traditional parallel strategies (such 
as tensor parallelism and sequence parallelism) and mechanized asynchronous communication 
mechanisms. Although these methods can achieve basic parallelization, they have two drawbacks. 
On one hand, the dependence on high - computing - power GPU clusters leads to high hardware 
costs. On the other hand, frequent cross - device communication introduces additional overhead 
of more than 30\%.
To address this issue, this paper proposes an adaptive communication optimization method based 
on feature similarity. By quantitatively analyzing the similarity of features between adjacent 
diffusion steps, a communication strategy for dynamically freezing key - value (KV) transmission 
is designed, and an asynchronous transmission mode is adopted to reduce communication overhead. 
Experimental results show that, under the premise of ensuring image - generation quality, 
this method significantly reduces distributed inference latency (up to 33\% at most), providing
 a solution for the efficient deployment of distributed inference of diffusion models on 
 low - cost commercial GPU clusters.



\noindent \textbf{Key words:} Diffusion models、 Distributed inference、 
Key-value (KV) cache、Communication optimization
