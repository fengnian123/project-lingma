\newpage

\section{绪论}
\subsection{论文研究背景及研究意义}
\zihao{-4} 
\fangsong
\par
近年来,基于扩散模型的生成式模型在计算机视觉领域展现出革命性的突破。
这类模型通过从简单噪声到复杂结果的逐步去噪机制(即正向扩散与反向去噪阶段),
在图像生成、编辑、视频生成及3D场景建模等任务中实现了高质量且稳定的输出\cite{ho2020denoising}。以2024年2月发布的Sora为例,
该视频生成模型凭借扩散模型的核心原理,在时空一致性建模上展现出突破性的表现\cite{brooks2024video}。
同时,Stable Diffusion等文生图模型则已经应用到创意设计、影视特效及交互式场景构建等领域
\cite{lyu2019advances,dhariwal2021diffusion,rombach2022high}。
\par
然而,扩散模型的迭代生成特性使其面临推理性能的挑战:由于每一步骤迭代都需要依赖前序状态,
且推理过程需经历大量迭代,导致生成效率受限并且对硬件算力需求较高\cite{Li2023SnapFusionTD}。
针对这个问题,很多研究者尝试通过减少推理步数、优化缓存策略等方式提升性能\cite{song2020denoising, liu2024difffno,Ma2023DeepCacheAD, salimans2022progressive},
也有研究者采用分布式推理方法使用GPU集群进行推理加速\cite{chen2024asyncdiff, li2024distrifusion}
但分布式推理方法在视觉内容生成场景中仍存在
根本性矛盾——图像、视频等全局一致性的要求使得传统并行计算方式很难直接应用。当前主流方案虽然可以通过多GPU协
作加速计算,但需要引入很高的通信代价:为了保证画面各区域的一致性,推理框架必须进行频繁的数据同步,
这不仅抵消了很大一部分并行计算优势,还要求部署在NVLink等高端硬件架构,提高了应用门槛与成本\cite{li2024distrifusion}。
\par
面对这一困境,本文深入研究扩散模型推理过程中的数据特征及变化规律,
提出了一种结合通信冻结与异步传输机制的优化框架。通过识别并减少冗余通信、将通信操作嵌入
计算过程,有效降低了跨设备数据的通信开销,使得该方法在低成本商用 GPU
集群上即可实现接近高算力GPU集群的推理性能,在保证扩散模型生成质量优势的同时,降低了硬件需求门槛,
为大规模视觉生成任务提供了兼具效率与成本的解决方案。
\subsection{论文主要研究内容}
\par
本研究围绕扩散模型的分布式推理优化展开,重点关注其核心架构特性与计算-通信异步机制。
通过系统分析基于扩散模型机制的典型生成式模型(以Stable Diffusion为代表)
的推理运行机制,结合硬件资源特性与算法进行设计创新,研究内容主要包括以下六个方面:
\begin{itemize}
    \item {\bfseries 扩散模型架构解析:}以Stable Diffusion为例,
    深入研究其正向扩散与反向去噪阶段的核心模块(如Unet噪声预测网络、self-attention注意力机制\cite{dhariwal2021diffusion}等)。
    量化分析各模块的参数规模与变化特征,揭示扩散模型迭代过程中关键变量的动态变化趋势,
    为后续优化提供理论依据。
    \item {\bfseries 传统分布式推理的通信机制分析:}针对主要分布式推理框架(如张量并行与模型并行),
    系统研究其在扩散模型场景下的通信机制。重点研究不同层间参数传递量、通信开销占比以及异步通信机制。
    \item {\bfseries 硬件通信调研:}对比NVLink互联架构与PCIE GPU集群的关键指标差异以及部署方式,
    研究使用低成本商用GPU的通信速率瓶颈,为后续算法设计的硬件支持奠定基础\cite{A100, RTX4090}。
    \item {\bfseries PyTorch通信方法:}针对PyTorch分布式工具包(DistributedDataParallel),
    研究all\_reduce、all\_gather、broadcast等方法在扩散模型分布式推理场景下的适用性,为后续算法设计
    提供实际应用方法\cite{PytorchDistributedDataParallel}。
    \item {\bfseries 通信冻结机制及异步通信模块设计:}研究基于特征相似度的自适应通信冻结策略,
    将通信延迟尽可能嵌入计算流水线,达到计算-通信重叠的效果。
    \item {\bfseries 性能验证与适用性分析:}研究构建标准化的测试框架,选取典型的图像数据集以及
    图像质量指标进行对比实验,分析本方法的的优化性能,在保证图像
    质量以及适用性的基础上,减小通信开销,降低成本。
\end{itemize}

\subsection{论文组织方式}
\par
本论文总体组织方式如下:
\begin{itemize}
    \item {\bfseries 第一章\ 绪论部分:}明确研究动机,从扩散模型的发展开始,
    阐述扩散模型在视觉生成领域的优势,并指出其性能瓶颈以及传统分布式推理方案的局限性,
    从而提出面向商用低成本GPU集群的通信优化框架。
    \item {\bfseries 第二章\ 背景知识和相关技术:}阐述后续研究用到的扩散模型原理,主要分析
    各模块作用以及在通信中需要传递的数据部分;详细分析目前主要的分布式推理方案(如distrifusion)运行机制及瓶颈问题;
    研究NVLink与PCIE集群的性能及成本;分析all\_reduce、all\_gather、broadcast等方法
    的具体通信方式。
    \item {\bfseries 第三章\ 基础研究工作:}通过测试代码跟踪扩散模型分布式推理过程,
    统计各层间参数传递量(如self-attention、cross-attention层权重等);通过L2距离
    等特征相似度计算方法,量化相邻迭代步之间的通信冗余程度;研究distrifusion等框架的具体通信机制,
    量化通信延迟及通信延迟与通信开销占比,以便与本文方法进行后续优化和对比
    \item {\bfseries 第四章\ 优化算法设计:}基于已有的研究数据结果,提出优化创新方案,采取相似度
    检测与通信冻结策略,设计通信冻结模块,同时设计调整通信操作以嵌入计算流水线的方案。
    \item {\bfseries 第五章\ 测试及结果分析:}使用A100 NVLink集群、RTX 4090 集群进行测试,
    使用多种方案进行对比(如标准扩散模型、传统分布式扩散模型、本论文基础方案、本论文优化方案等),
    展示通信开销、总体延迟等数据结果,并展示在典型数据集上的生成质量对比。
    \item {\bfseries 第六章\ 总结与展望:}总结本文总体优化效果及使用的框架机制,同时总结本论文
    存在的不足以及该领域未来可能的优化方向,最后对本人工作进行进一步展望,提出后续可能继续研究
    的方向。
\end{itemize}
